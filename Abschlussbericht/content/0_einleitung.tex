\chapter{Einleitung} \label{ch:einleitung}

Im Rahmen dieses IT-Projekts wurde das System CYBRAIL entwickelt, um die Integrität und Transparenz akademischer Prüfungen signifikant zu verbessern. Mit der fortschreitenden Digitalisierung von Prüfungsprozessen gewinnt der Einsatz von Tools wie dem Safe Exam Browser (SEB) zunehmend an Bedeutung, da sie gewährleisten, dass Prüfungen unter fairen Bedingungen durchgeführt werden und unerlaubte Hilfsmittel ausgeschlossen bleiben. Allerdings stellt die Auswertung der durch den SEB generierten Protokolle (Logs) bislang eine erhebliche Herausforderung dar, insbesondere wenn es darum geht, potenziell verdächtige Aktivitäten in Echtzeit zu identifizieren und adäquat zu reagieren.

Bisher wurde der SEB primär zur Abschreckung von Studierenden verwendet, ohne die generierten Logs systematisch auszuwerten oder zu überprüfen, ob Studierende Wege gefunden haben, die Sicherheitsvorkehrungen des SEB zu umgehen. Diese Lücke birgt das Risiko, dass unrechtmäßige Aktivitäten unentdeckt bleiben und somit die Integrität der Prüfungsergebnisse gefährdet wird.

Das zentrale Ziel von CYBRAIL ist es, diesen Mangel durch die Automatisierung des Prüfungsprozesses zu beheben und den verantwortlichen Professoren eine effiziente, möglichst in Echtzeit ablaufende Kontrolle und Auswertung der SEB-Logs zu ermöglichen. Dies umfasst die Entwicklung von Algorithmen zur automatisierten Analyse und Klassifizierung der Log-Daten, um Verdachtsfälle unerlaubter Mittel schnell und präzise zu erkennen und gleichzeitig entsprechende Beweismittel bereitzustellen. Ein weiteres wichtiges Merkmal von CYBRAIL ist die Bereitstellung einer modularen Systemarchitektur mit flexiblen Schnittstellen, die es ermöglicht, weitere Prüfmechanismen bei Bedarf problemlos zu integrieren.

Durch diese Lösung sollen sowohl die Zuverlässigkeit und Transparenz des Prüfungsprozesses als auch die Nachvollziehbarkeit der Prüfungsergebnisse erheblich gesteigert werden.