\chapter{Kommunikation und Qualitätssicherung} \label{ch:kommunikation}
Um eine gute Kommunikation zu gewährleisten, haben wir uns entschieden, alle 2–3 Wochen ein Meeting auf agile Weise abzuhalten. 
In diesen Meetings wurde jeweils ein \gls{mvp} vorgestellt und besprochen, was im nächsten Sprint bis zum folgenden Meeting umgesetzt werden sollte.

Zur Unterstützung dieser agilen Arbeitsweise haben wir eine eigene OpenProject-Instanz gehostet, mit der wir unsere Workpackages (Epics, User Stories, Tasks, Bugs, etc.) verwalteten. 
Zudem wurden hier Notizen zur Vor- und Nachbereitung der Meetings geführt, um eine klare Dokumentation sicherzustellen.

Durch diese regelmäßige Kommunikation und die Vorstellung des aktuellen Standes konnten wir eine hohe Qualität gewährleisten und sinnvolle Architekturen sowie Programmier-Patterns von Anfang an planen oder durch Refactoring verbessern, um eine übersichtliche Code-Basis zu erhalten.
