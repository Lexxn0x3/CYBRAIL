\chapter{Projektziele} \label{ch:projektziele}
Wir als Team haben uns mehrere Ziele für dieses Projekt gesetzt, um ein solches Tool zu entwickeln.

\section{Log- und Informationssammlung}
Unser erstes Ziel war es herauszufinden, inwiefern wir Zugriff auf die gespeicherten Logs erhalten können, wie diese übertragen werden und in welchem Format sie vorliegen. 
Darüber hinaus haben wir untersucht, wie diese Informationen von einem Computer aufbereitet werden können, um sie anschließend zu analysieren.

\section{Demonstration der Erkennung von Auffälligkeiten}
Ein weiteres wichtiges Ziel war es, herauszufinden, welche Betrugsversuche überhaupt erkannt werden können und welche davon relevant sind. 
Ebenso war es von Bedeutung zu klären, welche Informationen für die Erkennung benötigt werden. 
Darüber hinaus mussten wir identifizieren, welche Betrugsversuche vermutlich nicht erkannt werden können und welche zusätzlichen Informationen oder Logs dafür erforderlich wären.\\
\\
Letztendlich galt es, zu bewerten, auf welche Betrugserkennungsmethoden wir uns im Rahmen dieses Projekts fokussieren wollen, um diese zu demonstrieren.

\section{Benutzerfreundlichkeit und einfache Bereitstellung}
Das letzte Ziel des Projekts war es, eine möglichst einfache Benutzeroberfläche sowie eine unkomplizierte Einrichtung zu ermöglichen. 
Unser Projekt sollte schnell und einfach in der Praxis einsetzbar sein, sei es auf einem Server oder einem PC, ohne dass ein komplexes Setup erforderlich ist.
