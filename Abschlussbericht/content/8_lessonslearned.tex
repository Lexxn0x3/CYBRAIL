In diesem Kapitel werden die wichtigsten Erkenntnisse aus dem Projekt zusammengefasst. 
Diese betreffen sowohl die Kommunikation im Team als auch das Management und technische Herausforderungen, die im Verlauf des Projekts aufgetreten sind.

\section{Kommunikation} 
Die Kommunikation mit dem Product Owner und den Kollegen ist entscheidend, um Problemstellen effizient zu lösen. 
Unterschiedliche Sichtweisen haben oft dazu beigetragen, eine gemeinsame Lösung zu finden, die sich aus mehreren Teilansätzen und Ideen zusammensetzte.

\section{Vereinbarung von Terminen} 
Eine klare Vereinbarung von Meeting-Terminen und Deadlines hat nach anfänglichen Schwierigkeiten dazu geführt, dass alle Teammitglieder pünktlich an den Besprechungen teilnehmen konnten.

\section{Strukturiertes Projektmanagement} 
Eine klare Definition von User Stories und Tasks hat maßgeblich dazu beigetragen, die Zusammenarbeit zu verbessern. 
Dies stand im deutlichen Kontrast zu unstrukturiertem Arbeiten, bei dem Teammitglieder unkoordiniert implementierten.

\section{Remote-Prüfungen} 
Das gesamte Thema der Remote-Prüfungen hat sich als äußerst anspruchsvoll herausgestellt, insbesondere in Bezug auf die Frage, wie Betrug effektiv verhindert werden kann. 
Am Ende bleibt kein System vollkommen sicher, insbesondere wenn die Klausurteilnehmer Informatik studieren – nach dem Motto „alles ist Open Source, wenn du Assembly verstehst“. 
Letztendlich kann jedes System so modifiziert werden, dass es den Anschein erweckt, als sei alles rechtmäßig abgelaufen. 
Die einzige Möglichkeit besteht darin, die Hürde für Betrug so hoch zu setzen, dass nur wenige Studierende sie überwinden können – ähnlich wie bei Präsenzklausuren, wo Betrug ebenfalls nicht vollständig ausgeschlossen werden kann.
