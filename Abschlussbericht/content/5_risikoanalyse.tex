\chapter{Risikoanalyse und Risikomanagement} \label{ch:Risikoanalyse}
Bei der Entwicklung von \gls{cybrail} wurden verschiedene Risiken identifiziert, um mögliche Hindernisse oder Probleme während des Projekts frühzeitig zu erkennen und Gegenmaßnahmen zu definieren. 
Die Risiken lassen sich in fünf Hauptkategorien unterteilen:

\begin{itemize} \item \textbf{Technische Risiken:} Diese umfassen Softwarefehler, Kompatibilitätsprobleme zwischen verschiedenen Versionen des Safe Exam Browsers (SEB) und Sicherheitslücken. 
\item \textbf{Datenschutz- und Compliance-Risiken:} Hier geht es vor allem um Datenschutzverletzungen und die Einhaltung von Vorschriften zum Schutz der Daten der Studierenden. 
\item \textbf{Betriebliche Risiken:} Diese betreffen Ausfälle oder Unterbrechungen des Systems sowie Skalierbarkeitsprobleme. 
\item \textbf{Annahme- und Akzeptanzrisiken:} Hierzu gehören der Widerstand von Lehrenden und Studierenden sowie der Mangel an Schulungen und Unterstützung. 
\item \textbf{Projektspezifische Risiken:} Diese umfassen den fehlenden Zugriff auf Logdaten der Studierenden und die Anbindung an universitäre Dienste wie Moodle nach der Implementierung. \end{itemize}

Basierend auf der Wahrscheinlichkeit und dem Schadensausmaß jedes Risikos wurde eine Risikoprioritätszahl (RPZ) berechnet. 
Diese gibt an, wie dringend das Risiko behandelt werden muss. Für jedes Risiko wurden außerdem Gegenmaßnahmen entwickelt, um dessen Auswirkungen zu minimieren oder ganz zu vermeiden.

Die folgende Tabelle fasst die wichtigsten Risiken zusammen, die im Projekt identifiziert wurden:

\begin{landscape}
\begin{table}[htbp]
\centering
\begin{tabular}{|p{7cm}|c|c|c|p{6cm}|}
\hline
\textbf{Risiko} & \textbf{Schadenshöhe (1–10)} & \textbf{Wahrscheinlichkeit} & \textbf{RPZ} & \textbf{Gegenmaßnahmen} \\
\hline
Softwarefehler und -probleme: Wie man sie handhabt und was nach dem Projekt passiert & 2 & 0,7 & 1,4 & FAQ, Tests, abhängig von der Geschichte \\
\hline
Kompatibilitätsprobleme mit verschiedenen Versionen des SEB & 8 & 0,1 & 0,8 & Konfigurierbar \\
\hline
Sicherheitslücken & 7 & 0,5 & 3,5 & Sicherheitskonzepte, Penetrationstests \\
\hline
Datenschutzverletzungen & 8 & 0,4 & 3,2 & Sicherheitskonzepte, Penetrationstests, Datenspeicherlebensdauer \\
\hline
Einhaltung der Datenschutzvorschriften der Studierenden & 7 & 0,2 & 1,4 & Datensparsamkeit \\
\hline
Ausfallzeiten und Unterbrechungen & 2 & 0,2 & 0,4 & -- \\
\hline
Skalierungsprobleme & 2 & 0,2 & 0,4 & -- \\
\hline
Widerstand von Lehrenden und Studierenden & 2 & 0,6 & 1,2 & -- \\
\hline
Mangel an Schulungen und Unterstützung & 8 & 0,3 & 2,4 & Dokumentation, Benutzeroberfläche \\
\hline
Kein Zugriff auf Logdaten der Studierenden & 10 & 0,25 & 2,5 & -- \\
\hline
Kein Zugriff auf universitäre Dienste wie Moodle nach der Implementierung & 10 & 0,1 & 1 & -- \\
\hline
\end{tabular}
\caption{Risikobewertung und Gegenmaßnahmen}
\end{table}
\end{landscape}

\section{Zusammenfassung der Risikobewertung}

Die Risikobewertung zeigt, dass die kritischsten Risiken in den Bereichen Datenschutzverletzungen und Sicherheitslücken liegen, die beide durch entsprechende Sicherheitsmaßnahmen und Penetrationstests gemindert werden können. 
Gravierend wäre jedoch der fehlende Zugriff auf die Logdaten der Studierenden, da diese essenziell für die Funktion unseres Tools sind. 
Ohne diese Daten wäre eine Auswertung und Erkennung von Betrugsversuchen nicht möglich. 
Allerdings schätzen wir die Wahrscheinlichkeit, dass hierfür keine Lösung gefunden werden kann, als sehr gering ein. 
Mit den entsprechenden technischen und organisatorischen Maßnahmen, wie der Nutzung des \gls{seb}-Servers und der korrekten Konfiguration des Systems, sollten wir in der Lage sein, auf die notwendigen Logs zuzugreifen und diese sicher zu analysieren.
