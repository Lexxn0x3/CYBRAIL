\chapter{Ergebnisse und Leistungen} \label{ch:ergebnisse}
Als Gesamtergebnis haben wir nun einen Docker-Server und eine Client-App entwickelt, die alle Logdateien des \gls{seb} senden und Timing-Informationen der Eingaben der Studierenden speichern kann, um diese später auf ihre Legitimität hin zu überprüfen.\\
\\
Der Docker-Server verfügt über ein eigenes Webinterface, mit dem die Auswertung der Logdateien einer Klausur für alle Studierenden verwaltet werden kann. 
Alle Logdateien der Studierenden werden an den Server gesendet und dort gesammelt. 
Nach der Zuordnung der Logs zu den jeweiligen Studierenden wird ein modularer Analyseprozess gestartet. 
Das Programm kann beliebige Module aufrufen, die dem Standard entsprechen, und sammelt von jedem dieser Module Ergebnisse für jeden Studierenden. 
Sollte ein neues Modul erstellt oder Anpassungen vorgenommen werden müssen, kann dies einfach ohne Neukompilierung geschehen.\\
\\
Ähnliches gilt für die Konfigurationen der einzelnen Module. 
Jede Konfiguration ist ebenfalls generisch, und das \gls{ui} zur Konfiguration der Module setzt sich automatisch aus den vorhandenen Konfigurationsdateien zusammen.\\
\\
Auch das Parsing der Logs funktioniert ähnlich. Mithilfe eines Strategy-Patterns können neue Logtypen einfach integriert und in einem Modul verarbeitet werden.\\
\\
Aktuell stehen folgende Module zur Verfügung, die:\\
\\
\begin{itemize} 
  \item Copy-Paste-ähnliches Autotyping erkennen können 
  \item Auffälligkeiten in der Displaykonfiguration (z.B. zu viele Displays oder Änderungen der Auflösung) erkennen können 
  \item Auffälligkeiten bei Neuinitialisierungen des \gls{seb} feststellen können 
  \item Die Integrität des \gls{seb} prüfen können 
  \item Eine VM erkennen können, wenn der \gls{seb} innerhalb einer VM gestartet wird, was normalerweise durch den \gls{seb} verhindert wird   
  \item Ungewöhnliche Netzwerkaktivitäten feststellen können 
  \item Ungewöhnliches Shutdown-Verhalten des \gls{seb} erkennen können 
\end{itemize}

Zusätzlich zu diesen Modulen steht eine Python-Bibliothek zur Verfügung, die wiederkehrende Methoden in den Modulen abstrahiert.
