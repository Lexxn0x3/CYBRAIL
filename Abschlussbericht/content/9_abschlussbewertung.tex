\chapter{Fazit und Ausblick} \label{ch:fazit}
Abschließend lässt sich sagen, dass unser Projekt nahezu alle Ziele erreicht hat. 
Wir konnten erfolgreich zeigen, dass es möglich ist, automatisch die Logdateien von Studierenden auszuwerten und diese auf Auffälligkeiten zu untersuchen.\\
\\
All dies geschah in einem sehr flexiblen Framework, das sich an veränderte Gegebenheiten anpassen kann.\\
\\
Es bleibt jedoch fraglich, ob ein solches oder ähnliches System in der Praxis eingesetzt werden darf. 
Letztlich verfügt der \gls{seb} mit dem \gls{seb}-Server bereits über eine eigene Möglichkeit, Logdateien an einen zentralen Server zu übermitteln.\\
\\
Es besteht jedoch großes Potenzial, weitere Testmodule zu entwickeln und diese auf ein praxisnahes Szenario anzupassen. 
Es gab viele Vorschläge, was man zusätzlich in den Logs erkennen könnte. 
Dank des generischen Ansatzes der gesamten Architektur ist es durchaus möglich, das Programm in Zukunft wieder aufzugreifen und weiterzuentwickeln.

\chapter*{Disclaimer} 
Dieses Projekt wurde im Rahmen einer Machbarkeitsstudie entwickelt, um zu prüfen, ob es technisch möglich ist, die Logdateien von Studierenden automatisch auszuwerten. 
Das Programm spiegelt in seinem aktuellen Zustand den experimentellen Charakter dieser Studie wider und ist nicht für den produktiven Einsatz vorgesehen.\\
\\
Der Quellcode ist unter folgendem Link verfügbar: \href{https://github.com/Lexxn0x3/CYBRAIL}{https://github.com/Lexxn0x3/CYBRAIL}. 
Das Projekt steht unter der \texttt{GPL-3.0}-Lizenz, die ebenfalls im Repository zu finden ist.
